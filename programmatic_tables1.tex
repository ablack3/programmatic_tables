\documentclass[]{article}
\usepackage{lmodern}
\usepackage{amssymb,amsmath}
\usepackage{ifxetex,ifluatex}
\usepackage{fixltx2e} % provides \textsubscript
\ifnum 0\ifxetex 1\fi\ifluatex 1\fi=0 % if pdftex
  \usepackage[T1]{fontenc}
  \usepackage[utf8]{inputenc}
\else % if luatex or xelatex
  \ifxetex
    \usepackage{mathspec}
  \else
    \usepackage{fontspec}
  \fi
  \defaultfontfeatures{Ligatures=TeX,Scale=MatchLowercase}
\fi
% use upquote if available, for straight quotes in verbatim environments
\IfFileExists{upquote.sty}{\usepackage{upquote}}{}
% use microtype if available
\IfFileExists{microtype.sty}{%
\usepackage{microtype}
\UseMicrotypeSet[protrusion]{basicmath} % disable protrusion for tt fonts
}{}
\usepackage[margin=1in]{geometry}
\usepackage{hyperref}
\hypersetup{unicode=true,
            pdftitle={Programmatic Tables 1},
            pdfborder={0 0 0},
            breaklinks=true}
\urlstyle{same}  % don't use monospace font for urls
\usepackage{color}
\usepackage{fancyvrb}
\newcommand{\VerbBar}{|}
\newcommand{\VERB}{\Verb[commandchars=\\\{\}]}
\DefineVerbatimEnvironment{Highlighting}{Verbatim}{commandchars=\\\{\}}
% Add ',fontsize=\small' for more characters per line
\usepackage{framed}
\definecolor{shadecolor}{RGB}{248,248,248}
\newenvironment{Shaded}{\begin{snugshade}}{\end{snugshade}}
\newcommand{\KeywordTok}[1]{\textcolor[rgb]{0.13,0.29,0.53}{\textbf{#1}}}
\newcommand{\DataTypeTok}[1]{\textcolor[rgb]{0.13,0.29,0.53}{#1}}
\newcommand{\DecValTok}[1]{\textcolor[rgb]{0.00,0.00,0.81}{#1}}
\newcommand{\BaseNTok}[1]{\textcolor[rgb]{0.00,0.00,0.81}{#1}}
\newcommand{\FloatTok}[1]{\textcolor[rgb]{0.00,0.00,0.81}{#1}}
\newcommand{\ConstantTok}[1]{\textcolor[rgb]{0.00,0.00,0.00}{#1}}
\newcommand{\CharTok}[1]{\textcolor[rgb]{0.31,0.60,0.02}{#1}}
\newcommand{\SpecialCharTok}[1]{\textcolor[rgb]{0.00,0.00,0.00}{#1}}
\newcommand{\StringTok}[1]{\textcolor[rgb]{0.31,0.60,0.02}{#1}}
\newcommand{\VerbatimStringTok}[1]{\textcolor[rgb]{0.31,0.60,0.02}{#1}}
\newcommand{\SpecialStringTok}[1]{\textcolor[rgb]{0.31,0.60,0.02}{#1}}
\newcommand{\ImportTok}[1]{#1}
\newcommand{\CommentTok}[1]{\textcolor[rgb]{0.56,0.35,0.01}{\textit{#1}}}
\newcommand{\DocumentationTok}[1]{\textcolor[rgb]{0.56,0.35,0.01}{\textbf{\textit{#1}}}}
\newcommand{\AnnotationTok}[1]{\textcolor[rgb]{0.56,0.35,0.01}{\textbf{\textit{#1}}}}
\newcommand{\CommentVarTok}[1]{\textcolor[rgb]{0.56,0.35,0.01}{\textbf{\textit{#1}}}}
\newcommand{\OtherTok}[1]{\textcolor[rgb]{0.56,0.35,0.01}{#1}}
\newcommand{\FunctionTok}[1]{\textcolor[rgb]{0.00,0.00,0.00}{#1}}
\newcommand{\VariableTok}[1]{\textcolor[rgb]{0.00,0.00,0.00}{#1}}
\newcommand{\ControlFlowTok}[1]{\textcolor[rgb]{0.13,0.29,0.53}{\textbf{#1}}}
\newcommand{\OperatorTok}[1]{\textcolor[rgb]{0.81,0.36,0.00}{\textbf{#1}}}
\newcommand{\BuiltInTok}[1]{#1}
\newcommand{\ExtensionTok}[1]{#1}
\newcommand{\PreprocessorTok}[1]{\textcolor[rgb]{0.56,0.35,0.01}{\textit{#1}}}
\newcommand{\AttributeTok}[1]{\textcolor[rgb]{0.77,0.63,0.00}{#1}}
\newcommand{\RegionMarkerTok}[1]{#1}
\newcommand{\InformationTok}[1]{\textcolor[rgb]{0.56,0.35,0.01}{\textbf{\textit{#1}}}}
\newcommand{\WarningTok}[1]{\textcolor[rgb]{0.56,0.35,0.01}{\textbf{\textit{#1}}}}
\newcommand{\AlertTok}[1]{\textcolor[rgb]{0.94,0.16,0.16}{#1}}
\newcommand{\ErrorTok}[1]{\textcolor[rgb]{0.64,0.00,0.00}{\textbf{#1}}}
\newcommand{\NormalTok}[1]{#1}
\usepackage{graphicx,grffile}
\makeatletter
\def\maxwidth{\ifdim\Gin@nat@width>\linewidth\linewidth\else\Gin@nat@width\fi}
\def\maxheight{\ifdim\Gin@nat@height>\textheight\textheight\else\Gin@nat@height\fi}
\makeatother
% Scale images if necessary, so that they will not overflow the page
% margins by default, and it is still possible to overwrite the defaults
% using explicit options in \includegraphics[width, height, ...]{}
\setkeys{Gin}{width=\maxwidth,height=\maxheight,keepaspectratio}
\IfFileExists{parskip.sty}{%
\usepackage{parskip}
}{% else
\setlength{\parindent}{0pt}
\setlength{\parskip}{6pt plus 2pt minus 1pt}
}
\setlength{\emergencystretch}{3em}  % prevent overfull lines
\providecommand{\tightlist}{%
  \setlength{\itemsep}{0pt}\setlength{\parskip}{0pt}}
\setcounter{secnumdepth}{0}
% Redefines (sub)paragraphs to behave more like sections
\ifx\paragraph\undefined\else
\let\oldparagraph\paragraph
\renewcommand{\paragraph}[1]{\oldparagraph{#1}\mbox{}}
\fi
\ifx\subparagraph\undefined\else
\let\oldsubparagraph\subparagraph
\renewcommand{\subparagraph}[1]{\oldsubparagraph{#1}\mbox{}}
\fi

%%% Use protect on footnotes to avoid problems with footnotes in titles
\let\rmarkdownfootnote\footnote%
\def\footnote{\protect\rmarkdownfootnote}

%%% Change title format to be more compact
\usepackage{titling}

% Create subtitle command for use in maketitle
\newcommand{\subtitle}[1]{
  \posttitle{
    \begin{center}\large#1\end{center}
    }
}

\setlength{\droptitle}{-2em}

  \title{Programmatic Tables 1}
    \pretitle{\vspace{\droptitle}\centering\huge}
  \posttitle{\par}
    \author{}
    \preauthor{}\postauthor{}
    \date{}
    \predate{}\postdate{}
  
\usepackage{longtable}
\usepackage{booktabs}
\usepackage{caption}

\begin{document}
\maketitle

\textbf{Much of this worksheet is copied directly from the gt intro
vignette}

How should we populate tables for academic paper or presentation? This
is the fundamental question we are looking to address in the next two R
user group sessions. If you have ever created tables ``by hand'' then
you know how time consuming this process can be. It is also error prone
since we humans are not nearly as good at copying as computers.
Eventually we will make a copy error. Also updating a table that you
have created by hand can be a difficult process.

In this exercise we will introduce some alternative techniques using the
\texttt{gt} package in R.

The \textbf{gt} package is all about making it simple to produce
nice-looking display tables. Display tables? Well yes, we are trying to
distinguish between data tables (e.g., tibbles, \texttt{data.frame}s,
etc.) and those tables you'd find in a web page, a journal article, or
in a magazine. Such tables can likewise be called presentation tables,
summary tables, or just tables really. Here are some examples, ripped
straight from the web:

We can think of display tables as output only, where we'd not want to
use them as input ever again. Other features include annotations, table
element styling, and text transformations that serve to communicate the
subject matter more clearly.

\subsubsection{\texorpdfstring{A Walkthrough of the \textbf{gt} Basics
with a Simple
Table}{A Walkthrough of the gt Basics with a Simple Table}}\label{a-walkthrough-of-the-gt-basics-with-a-simple-table}

Let's use a less common dataset that is available in the R
\textbf{datasets} package: \texttt{islands}. It's actually not a data
frame but a named vector. That's okay though, we can use use
\textbf{dplyr} and prepare a tibble from it:

\begin{Shaded}
\begin{Highlighting}[]
\CommentTok{# Take the `islands` dataset and use some}
\CommentTok{# dplyr functionality to obtain the ten}
\CommentTok{# biggest islands in the world}
\NormalTok{islands_tbl <-}\StringTok{ }\KeywordTok{tibble}\NormalTok{(}\DataTypeTok{name =} \KeywordTok{names}\NormalTok{(islands), }\DataTypeTok{size =}\NormalTok{ islands) }\OperatorTok
\StringTok{  }\KeywordTok{arrange}\NormalTok{(}\KeywordTok{desc}\NormalTok{(size)) }\OperatorTok
\StringTok{  }\KeywordTok{slice}\NormalTok{(}\DecValTok{1}\OperatorTok{:}\DecValTok{10}\NormalTok{)}
\CommentTok{# Display the table}
\NormalTok{islands_tbl}
\CommentTok{#> # A tibble: 10 x 2}
\CommentTok{#>    name           size}
\CommentTok{#>    <chr>         <dbl>}
\CommentTok{#>  1 Asia          16988}
\CommentTok{#>  2 Africa        11506}
\CommentTok{#>  3 North America  9390}
\CommentTok{#>  4 South America  6795}
\CommentTok{#>  5 Antarctica     5500}
\CommentTok{#>  6 Europe         3745}
\CommentTok{#>  7 Australia      2968}
\CommentTok{#>  8 Greenland       840}
\CommentTok{#>  9 New Guinea      306}
\CommentTok{#> 10 Borneo          280}
\end{Highlighting}
\end{Shaded}

If we pass \texttt{islands\_tbl} to the \emph{function} \texttt{gt()},
we'll get a \textbf{gt Table} as output. As an aside, we could have
easily used a data frame instead as valid \textbf{Table Data} for
\textbf{gt}.

\begin{Shaded}
\begin{Highlighting}[]
\CommentTok{# Create a display table showing ten of}
\CommentTok{# the largest islands in the world}
\NormalTok{gt_tbl <-}\StringTok{ }\KeywordTok{gt}\NormalTok{(}\DataTypeTok{data =}\NormalTok{ islands_tbl)}
\CommentTok{# Show the gt Table}
\NormalTok{gt_tbl}
\end{Highlighting}
\end{Shaded}

\captionsetup[table]{labelformat=empty,skip=1pt}

\begin{longtable}{lr}
\toprule
name & size \\ 
\midrule
Asia & 16988 \\ 
Africa & 11506 \\ 
North America & 9390 \\ 
South America & 6795 \\ 
Antarctica & 5500 \\ 
Europe & 3745 \\ 
Australia & 2968 \\ 
Greenland & 840 \\ 
New Guinea & 306 \\ 
Borneo & 280 \\ 
\bottomrule
\end{longtable}

That doesn't look too bad. Sure, it's basic but we really didn't really
ask for much. We did receive a proper table with column labels and the
data. Also, that default striping is a nice touch. Oftentimes however,
you'll want a bit more: a \textbf{Table header}, a \textbf{Stub}, and
sometimes \emph{footnotes} and \emph{source notes} in the \textbf{Table
Footer} part.

\subsubsection{Adding Parts to this Simple
Table}\label{adding-parts-to-this-simple-table}

The \textbf{gt} package makes it relatively easy to add parts so that
the resulting \textbf{gt Table} better conveys the information you want
to present. These table parts work well together and there the possible
variations in arrangement can handle most tabular presentation needs.
The previous \textbf{gt Table} demonstrated had only two parts, the
\textbf{Column Labels} and the \textbf{Table Body}. The next few
examples will show all of the other table parts that are available.

This is the way the main parts of a table (and their subparts) fit
together:

The parts (roughly from top to bottom) are:

\begin{itemize}
\tightlist
\item
  the \textbf{Table Header} (optional; with a \textbf{title} and
  possibly a \textbf{subtitle})
\item
  the \textbf{Stub} and the \textbf{Stub Head} (optional; contains
  \emph{row labels}, optionally within \emph{row groups} having
  \emph{row group labels} and possibly \emph{summary labels} when a
  summary is present)
\item
  the \textbf{Column Labels} (contains \emph{column labels}, optionally
  under \emph{spanner column labels})
\item
  the \textbf{Table Body} (contains \emph{columns} and \emph{rows} of
  \emph{cells})
\item
  the \textbf{Table Footer} (optional; possibly with \textbf{footnotes}
  and \textbf{source notes})
\end{itemize}

The way that we add parts like the \textbf{Table Header} and
\emph{footnotes} in the \textbf{Table Footer} is to use the
\texttt{tab\_*()} family of functions. A \textbf{Table Header} is easy
to add so let's see how the previous table looks with a \textbf{title}
and a \textbf{subtitle}. We can add this part using the
\texttt{tab\_header()} function:

\begin{Shaded}
\begin{Highlighting}[]
\CommentTok{# Make a display table with the `islands_tbl`}
\CommentTok{# table; put a heading just above the column labels}
\NormalTok{gt_tbl <-}\StringTok{ }
\StringTok{  }\NormalTok{gt_tbl }\OperatorTok
\StringTok{  }\KeywordTok{tab_header}\NormalTok{(}
    \DataTypeTok{title =} \StringTok{"Large Landmasses of the World"}\NormalTok{,}
    \DataTypeTok{subtitle =} \StringTok{"The top ten largest are presented"}
\NormalTok{  )}
\CommentTok{# Show the gt Table}
\NormalTok{gt_tbl}
\end{Highlighting}
\end{Shaded}

\captionsetup[table]{labelformat=empty,skip=1pt}

\begin{longtable}{lr}
\caption*{
\large Large Landmasses of the World\\ 
\small The top ten largest are presented\\ 
} \\ 
\toprule
name & size \\ 
\midrule
Asia & 16988 \\ 
Africa & 11506 \\ 
North America & 9390 \\ 
South America & 6795 \\ 
Antarctica & 5500 \\ 
Europe & 3745 \\ 
Australia & 2968 \\ 
Greenland & 840 \\ 
New Guinea & 306 \\ 
Borneo & 280 \\ 
\bottomrule
\end{longtable}

The \textbf{Header} table part provides an opportunity to describe the
data that's presented. The \texttt{subtitle}, which functions as a
subtitle, is an optional part of the \textbf{Header}. We may also style
the \texttt{title} and \texttt{subtitle} using Markdown! We do this by
wrapping the values passed to \texttt{title} or \texttt{subtitle} with
the \texttt{md()} function. Here is an example with the table data
truncated for brevity:

\begin{Shaded}
\begin{Highlighting}[]
\CommentTok{# Use markdown for the heading's `title` and `subtitle` to}
\CommentTok{# add bold and italicized characters}
\KeywordTok{gt}\NormalTok{(islands_tbl[}\DecValTok{1}\OperatorTok{:}\DecValTok{2}\NormalTok{,]) }\OperatorTok
\StringTok{  }\KeywordTok{tab_header}\NormalTok{(}
    \DataTypeTok{title =} \KeywordTok{md}\NormalTok{(}\StringTok{"**Large Landmasses of the World**"}\NormalTok{),}
    \DataTypeTok{subtitle =} \KeywordTok{md}\NormalTok{(}\StringTok{"The *top two* largest are presented"}\NormalTok{)}
\NormalTok{  )}
\end{Highlighting}
\end{Shaded}

\captionsetup[table]{labelformat=empty,skip=1pt}

\begin{longtable}{lr}
\caption*{
\large \textbf{Large Landmasses of the World}\\ 
\small The \emph{top two} largest are presented\\ 
} \\ 
\toprule
name & size \\ 
\midrule
Asia & 16988 \\ 
Africa & 11506 \\ 
\bottomrule
\end{longtable}

A \textbf{source note} can be added to the table's \textbf{footer}
through use of the \texttt{tab\_source\_note()} function. It works in
the same way as \texttt{tab\_header()} (it also allows for Markdown
inputs) except it can be called multiple times---each invocation results
in the addition of a source note.

\begin{Shaded}
\begin{Highlighting}[]
\CommentTok{# Display the `islands_tbl` data with a heading and}
\CommentTok{# two source notes}
\NormalTok{gt_tbl <-}\StringTok{ }
\StringTok{  }\NormalTok{gt_tbl }\OperatorTok
\StringTok{  }\KeywordTok{tab_source_note}\NormalTok{(}
    \DataTypeTok{source_note =} \StringTok{"Source: The World Almanac and Book of Facts, 1975, page 406."}
\NormalTok{  ) }\OperatorTok
\StringTok{  }\KeywordTok{tab_source_note}\NormalTok{(}
    \DataTypeTok{source_note =} \KeywordTok{md}\NormalTok{(}\StringTok{"Reference: McNeil, D. R. (1977) *Interactive Data Analysis*. Wiley."}\NormalTok{)}
\NormalTok{  )}
\CommentTok{# Show the gt Table}
\NormalTok{gt_tbl}
\end{Highlighting}
\end{Shaded}

\captionsetup[table]{labelformat=empty,skip=1pt}

\begin{longtable}{lr}
\caption*{
\large Large Landmasses of the World\\ 
\small The top ten largest are presented\\ 
} \\ 
\toprule
name & size \\ 
\midrule
Asia & 16988 \\ 
Africa & 11506 \\ 
North America & 9390 \\ 
South America & 6795 \\ 
Antarctica & 5500 \\ 
Europe & 3745 \\ 
Australia & 2968 \\ 
Greenland & 840 \\ 
New Guinea & 306 \\ 
Borneo & 280 \\ 
\bottomrule
\end{longtable}\begin{minipage}{\linewidth}
Source: The World Almanac and Book of Facts, 1975, page 406.\\ 
Reference: McNeil, D. R. (1977) \emph{Interactive Data Analysis}. Wiley.\\ 
\end{minipage}

Footnotes live inside the \textbf{Footer} part and their reference
glyphs are attached to cell data. Footnotes are added with the
\texttt{tab\_footnote()} function. The helper function
\texttt{cells\_data()} can be used with the \texttt{location} argument
to specify which data cells should be the target of the footnote. The
\texttt{cells\_data()} helper has the two arguments \texttt{columns} and
\texttt{rows}. For each of these, we can supply (1) a vector of colnames
or rownames, (2) a vector of column/row indices, (3) bare column names
wrapped in \texttt{vars()} or row labels within \texttt{c()}, or (4) a
select helper function (\texttt{starts\_with()}, \texttt{ends\_with()},
\texttt{contains()}, \texttt{matches()}, \texttt{one\_of()}, and
\texttt{everything()}). For \texttt{rows} specifically, we can use a
conditional statement with column names as variables (e.g.,
\texttt{size\ \textgreater{}\ 15000}).

Here is a simple example on how a footnotes can be added to a table
cell. Let's add a footnote that references the \texttt{North\ America}
and \texttt{South\ America} cells in the \texttt{name} column:

\begin{Shaded}
\begin{Highlighting}[]
\CommentTok{# Add footnotes (the same text) to two different}
\CommentTok{# cell; data cells are targeted with `data_cells()`}
\NormalTok{gt_tbl <-}\StringTok{ }
\StringTok{  }\NormalTok{gt_tbl }\OperatorTok
\StringTok{  }\KeywordTok{tab_footnote}\NormalTok{(}
    \DataTypeTok{footnote =} \StringTok{"The Americas."}\NormalTok{,}
    \DataTypeTok{locations =} \KeywordTok{cells_data}\NormalTok{(}
      \DataTypeTok{columns =} \KeywordTok{vars}\NormalTok{(name),}
      \DataTypeTok{rows =} \DecValTok{3}\OperatorTok{:}\DecValTok{4}\NormalTok{)}
\NormalTok{  )}
\CommentTok{# Show the gt Table}
\NormalTok{gt_tbl}
\end{Highlighting}
\end{Shaded}

\captionsetup[table]{labelformat=empty,skip=1pt}

\begin{longtable}{lr}
\caption*{
\large Large Landmasses of the World\\ 
\small The top ten largest are presented\\ 
} \\ 
\toprule
name & size \\ 
\midrule
Asia & 16988 \\ 
Africa & 11506 \\ 
North America\textsuperscript{1} & 9390 \\ 
South America\textsuperscript{1} & 6795 \\ 
Antarctica & 5500 \\ 
Europe & 3745 \\ 
Australia & 2968 \\ 
Greenland & 840 \\ 
New Guinea & 306 \\ 
Borneo & 280 \\ 
\bottomrule
\end{longtable}

\vspace{-5mm}

\begin{minipage}{\linewidth}
\textsuperscript{1}The Americas. \\ 
\end{minipage}\begin{minipage}{\linewidth}
Source: The World Almanac and Book of Facts, 1975, page 406.\\ 
Reference: McNeil, D. R. (1977) \emph{Interactive Data Analysis}. Wiley.\\ 
\end{minipage}

Here is a slightly more complex example of adding footnotes that use
expressions in \texttt{rows} to help target cells in a column by the
underlying data in \texttt{islands\_tbl}. First, a set of \textbf{dplyr}
statements obtains the name of the `island' by largest landmass. This is
assigned to the \texttt{largest} object and is used in the first
\texttt{tab\_footnote()} call that targets the cell in the \texttt{size}
column that is next to a \texttt{name} value that is stored in
\texttt{largest} (`Asia'). The second \texttt{tab\_footnote()} is
similar except we are supplying a conditional statement that gets the
lowest population.

\begin{Shaded}
\begin{Highlighting}[]
\CommentTok{# Determine the row that contains the}
\CommentTok{# largest landmass ('Asia')}
\NormalTok{largest <-}\StringTok{ }
\StringTok{  }\NormalTok{islands_tbl }\OperatorTok\StringTok{ }
\StringTok{  }\KeywordTok{arrange}\NormalTok{(}\KeywordTok{desc}\NormalTok{(size)) }\OperatorTok
\StringTok{  }\KeywordTok{slice}\NormalTok{(}\DecValTok{1}\NormalTok{) }\OperatorTok
\StringTok{  }\KeywordTok{pull}\NormalTok{(name)}
\CommentTok{# Create two additional footnotes, using the}
\CommentTok{# `columns` and `where` arguments of `data_cells()`}
\NormalTok{gt_tbl <-}\StringTok{ }
\StringTok{  }\NormalTok{gt_tbl }\OperatorTok
\StringTok{  }\KeywordTok{tab_footnote}\NormalTok{(}
    \DataTypeTok{footnote =} \KeywordTok{md}\NormalTok{(}\StringTok{"The **largest** by area."}\NormalTok{),}
    \DataTypeTok{locations =} \KeywordTok{cells_data}\NormalTok{(}
      \DataTypeTok{columns =} \KeywordTok{vars}\NormalTok{(size),}
      \DataTypeTok{rows =}\NormalTok{ name }\OperatorTok{==}\StringTok{ }\NormalTok{largest)}
\NormalTok{  ) }\OperatorTok
\StringTok{  }\KeywordTok{tab_footnote}\NormalTok{(}
    \DataTypeTok{footnote =} \StringTok{"The lowest by population."}\NormalTok{,}
    \DataTypeTok{locations =} \KeywordTok{cells_data}\NormalTok{(}
      \DataTypeTok{columns =} \KeywordTok{vars}\NormalTok{(size),}
      \DataTypeTok{rows =}\NormalTok{ size }\OperatorTok{==}\StringTok{ }\KeywordTok{min}\NormalTok{(size))}
\NormalTok{  )}
\CommentTok{# Show the gt Table}
\NormalTok{gt_tbl}
\CommentTok{#> Warning: HTML tags found, and they will be removed.}
\CommentTok{#>  * set `options(gt.html_tag_check = FALSE)` to disable this check}
\end{Highlighting}
\end{Shaded}

\captionsetup[table]{labelformat=empty,skip=1pt}

\begin{longtable}{lr}
\caption*{
\large Large Landmasses of the World\\ 
\small The top ten largest are presented\\ 
} \\ 
\toprule
name & size \\ 
\midrule
Asia & 16988\textsuperscript{1} \\ 
Africa & 11506 \\ 
North America\textsuperscript{2} & 9390 \\ 
South America\textsuperscript{2} & 6795 \\ 
Antarctica & 5500 \\ 
Europe & 3745 \\ 
Australia & 2968 \\ 
Greenland & 840 \\ 
New Guinea & 306 \\ 
Borneo & 280\textsuperscript{3} \\ 
\bottomrule
\end{longtable}

\vspace{-5mm}

\begin{minipage}{\linewidth}
\textsuperscript{1}The largest by area. \\ 
\textsuperscript{2}The Americas. \\ 
\textsuperscript{3}The lowest by population. \\ 
\end{minipage}\begin{minipage}{\linewidth}
Source: The World Almanac and Book of Facts, 1975, page 406.\\ 
Reference: McNeil, D. R. (1977) \emph{Interactive Data Analysis}. Wiley.\\ 
\end{minipage}

We were able to supply the reference locations in the table by using the
\texttt{cells\_data()} helper function and supplying the necessary
targeting through the \texttt{columns} and \texttt{rows} arguments.
Other \texttt{cells\_*()} functions have similar interfaces and they
allow us to target cells in different parts of the table.

\subsubsection{The Stub}\label{the-stub}

The \textbf{Stub} is the area to the left in a table that contains
\emph{row labels}, and may contain \emph{row group labels}, and
\emph{summary labels}. Those subparts can be grouped in a sequence of
\emph{row groups}. The \textbf{Stub Head} provides a location for a
label that describes the \textbf{Stub}. The \textbf{Stub} is optional
since there are cases where a \textbf{Stub} wouldn't be useful (e.g.,
the display tables presented above were just fine without a
\textbf{Stub}).

An easy way to generate a \textbf{Stub} part is by specifying a stub
column in the \texttt{gt()} function with the \texttt{rowname\_col}
argument. Alternatively, we can have an input dataset with a column
named \texttt{rowname}---this magic column will signal to \textbf{gt}
that that column should be used as the stub, making \emph{row labels}.
Let's add a stub with our \texttt{islands\_tbl} dataset by modifying the
call to \texttt{gt()}:

\begin{Shaded}
\begin{Highlighting}[]
\CommentTok{# Create a gt table showing ten of the}
\CommentTok{# largest islands in the world; this}
\CommentTok{# time with a stub}
\NormalTok{gt_tbl <-}\StringTok{ }
\StringTok{  }\NormalTok{islands_tbl }\OperatorTok
\StringTok{  }\KeywordTok{gt}\NormalTok{(}\DataTypeTok{rowname_col =} \StringTok{"name"}\NormalTok{)}
\CommentTok{# Show the gt Table}
\NormalTok{gt_tbl}
\end{Highlighting}
\end{Shaded}

\captionsetup[table]{labelformat=empty,skip=1pt}

\begin{longtable}{lr}
\toprule
 & size \\ 
\midrule
Asia & 16988 \\ 
Africa & 11506 \\ 
North America & 9390 \\ 
South America & 6795 \\ 
Antarctica & 5500 \\ 
Europe & 3745 \\ 
Australia & 2968 \\ 
Greenland & 840 \\ 
New Guinea & 306 \\ 
Borneo & 280 \\ 
\bottomrule
\end{longtable}

Notice that the landmass names are off the the left in an unstriped
area? That's the \textbf{stub}. We can apply what's known as a
\textbf{stubhead label}. This label can be added with the
\texttt{tab\_stubhead\_label()} function:

\begin{Shaded}
\begin{Highlighting}[]
\CommentTok{# Generate a simple table with a stub}
\CommentTok{# and add a stubhead label}
\NormalTok{gt_tbl <-}\StringTok{ }
\StringTok{  }\NormalTok{gt_tbl }\OperatorTok
\StringTok{  }\KeywordTok{tab_stubhead_label}\NormalTok{(}\DataTypeTok{label =} \StringTok{"landmass"}\NormalTok{)}
\CommentTok{# Show the gt Table}
\NormalTok{gt_tbl}
\end{Highlighting}
\end{Shaded}

\captionsetup[table]{labelformat=empty,skip=1pt}

\begin{longtable}{lr}
\toprule
landmass & size \\ 
\midrule
Asia & 16988 \\ 
Africa & 11506 \\ 
North America & 9390 \\ 
South America & 6795 \\ 
Antarctica & 5500 \\ 
Europe & 3745 \\ 
Australia & 2968 \\ 
Greenland & 840 \\ 
New Guinea & 306 \\ 
Borneo & 280 \\ 
\bottomrule
\end{longtable}

A very important thing to note here is that the table now has one
column. Before, when there was no \textbf{stub}, two columns were
present (with \textbf{column labels} \texttt{name} and \texttt{size})
but now column number \texttt{1} (the only column) is \texttt{size}.

To apply our table parts as before (up to and including the footnotes)
we use the following statements:

\begin{Shaded}
\begin{Highlighting}[]
\CommentTok{# Display the `islands_tbl` data with a stub,}
\CommentTok{# a heading, source notes, and footnotes}
\NormalTok{gt_tbl <-}\StringTok{ }
\StringTok{  }\NormalTok{gt_tbl }\OperatorTok
\StringTok{  }\KeywordTok{tab_header}\NormalTok{(}
    \DataTypeTok{title =} \StringTok{"Large Landmasses of the World"}\NormalTok{,}
    \DataTypeTok{subtitle =} \StringTok{"The top ten largest are presented"}
\NormalTok{  ) }\OperatorTok
\StringTok{  }\KeywordTok{tab_source_note}\NormalTok{(}
    \DataTypeTok{source_note =} \StringTok{"Source: The World Almanac and Book of Facts, 1975, page 406."}
\NormalTok{  ) }\OperatorTok
\StringTok{  }\KeywordTok{tab_source_note}\NormalTok{(}
    \DataTypeTok{source_note =} \KeywordTok{md}\NormalTok{(}\StringTok{"Reference: McNeil, D. R. (1977) *Interactive Data Analysis*. Wiley."}\NormalTok{)}
\NormalTok{  ) }\OperatorTok
\StringTok{  }\KeywordTok{tab_footnote}\NormalTok{(}
    \DataTypeTok{footnote =} \KeywordTok{md}\NormalTok{(}\StringTok{"The **largest** by area."}\NormalTok{),}
    \DataTypeTok{locations =} \KeywordTok{cells_data}\NormalTok{(}
      \DataTypeTok{columns =} \KeywordTok{vars}\NormalTok{(size),}
      \DataTypeTok{rows =}\NormalTok{ largest)}
\NormalTok{  ) }\OperatorTok
\StringTok{  }\KeywordTok{tab_footnote}\NormalTok{(}
    \DataTypeTok{footnote =} \StringTok{"The lowest by population."}\NormalTok{,}
    \DataTypeTok{locations =} \KeywordTok{cells_data}\NormalTok{(}
      \DataTypeTok{columns =} \KeywordTok{vars}\NormalTok{(size),}
      \DataTypeTok{rows =} \KeywordTok{contains}\NormalTok{(}\StringTok{"arc"}\NormalTok{))}
\NormalTok{  )}
\CommentTok{# Show the gt Table}
\NormalTok{gt_tbl}
\CommentTok{#> Warning: HTML tags found, and they will be removed.}
\CommentTok{#>  * set `options(gt.html_tag_check = FALSE)` to disable this check}
\end{Highlighting}
\end{Shaded}

\captionsetup[table]{labelformat=empty,skip=1pt}

\begin{longtable}{lr}
\caption*{
\large Large Landmasses of the World\\ 
\small The top ten largest are presented\\ 
} \\ 
\toprule
landmass & size \\ 
\midrule
Asia & 16988\textsuperscript{1} \\ 
Africa & 11506 \\ 
North America & 9390 \\ 
South America & 6795 \\ 
Antarctica & 5500\textsuperscript{2} \\ 
Europe & 3745 \\ 
Australia & 2968 \\ 
Greenland & 840 \\ 
New Guinea & 306 \\ 
Borneo & 280 \\ 
\bottomrule
\end{longtable}

\vspace{-5mm}

\begin{minipage}{\linewidth}
\textsuperscript{1}The largest by area. \\ 
\textsuperscript{2}The lowest by population. \\ 
\end{minipage}\begin{minipage}{\linewidth}
Source: The World Almanac and Book of Facts, 1975, page 406.\\ 
Reference: McNeil, D. R. (1977) \emph{Interactive Data Analysis}. Wiley.\\ 
\end{minipage}

Let's incorporate row groups into the display table. This divides rows
into groups, creating \emph{row groups}, and results in a display of a
\emph{row group labels} right above the each group. This can be easily
done with a table containing row labels. We can make a new \emph{row
group} with each call of the \texttt{tab\_row\_group()} function. The
inputs are group names in the \texttt{group} argument, and row
references in the \texttt{rows} argument. We can use any of the
strategies to reference rows as we did we footnotes (e.g., vectors of
names/indices, select helpers, etc.).

Here we will create three row groups (with row group labels
\texttt{continent}, \texttt{country}, and \texttt{subregion}) to have a
grouping of rows.

\begin{Shaded}
\begin{Highlighting}[]
\CommentTok{# Create three row groups with the}
\CommentTok{# `tab_row_group()` function}
\NormalTok{gt_tbl <-}\StringTok{ }
\StringTok{  }\NormalTok{gt_tbl }\OperatorTok\StringTok{ }
\StringTok{  }\KeywordTok{tab_row_group}\NormalTok{(}
    \DataTypeTok{group =} \StringTok{"continent"}\NormalTok{,}
    \DataTypeTok{rows =} \DecValTok{1}\OperatorTok{:}\DecValTok{6}
\NormalTok{  ) }\OperatorTok
\StringTok{  }\KeywordTok{tab_row_group}\NormalTok{(}
    \DataTypeTok{group =} \StringTok{"country"}\NormalTok{,}
    \DataTypeTok{rows =} \KeywordTok{c}\NormalTok{(}\StringTok{"Australia"}\NormalTok{, }\StringTok{"Greenland"}\NormalTok{)}
\NormalTok{  ) }\OperatorTok
\StringTok{  }\KeywordTok{tab_row_group}\NormalTok{(}
    \DataTypeTok{group =} \StringTok{"subregion"}\NormalTok{,}
    \DataTypeTok{rows =} \KeywordTok{c}\NormalTok{(}\StringTok{"New Guinea"}\NormalTok{, }\StringTok{"Borneo"}\NormalTok{)}
\NormalTok{  )}
\CommentTok{# Show the gt Table}
\NormalTok{gt_tbl}
\CommentTok{#> Warning in min(rows_matched): no non-missing arguments to min; returning}
\CommentTok{#> Inf}
\CommentTok{#> Warning in max(rows_matched): no non-missing arguments to max; returning -}
\CommentTok{#> Inf}
\CommentTok{#> Warning: HTML tags found, and they will be removed.}
\CommentTok{#>  * set `options(gt.html_tag_check = FALSE)` to disable this check}
\end{Highlighting}
\end{Shaded}

\captionsetup[table]{labelformat=empty,skip=1pt}

\begin{longtable}{lr}
\caption*{
\large Large Landmasses of the World\\ 
\small The top ten largest are presented\\ 
} \\ 
\toprule
landmass & size \\ 
\midrule
\multicolumn{1}{l}{continent} \\ 
\midrule
Asia & 16988\textsuperscript{1} \\ 
Africa & 11506 \\ 
North America & 9390 \\ 
South America & 6795 \\ 
Antarctica & 5500\textsuperscript{2} \\ 
Europe & 3745 \\ 
\midrule
\multicolumn{1}{l}{country} \\ 
\midrule
Australia & 2968 \\ 
Greenland & 840 \\ 
\midrule
\multicolumn{1}{l}{subregion} \\ 
\midrule
New Guinea & 306 \\ 
Borneo & 280 \\ 
\bottomrule
\end{longtable}

\vspace{-5mm}

\begin{minipage}{\linewidth}
\textsuperscript{1}The largest by area. \\ 
\textsuperscript{2}The lowest by population. \\ 
\end{minipage}\begin{minipage}{\linewidth}
Source: The World Almanac and Book of Facts, 1975, page 406.\\ 
Reference: McNeil, D. R. (1977) \emph{Interactive Data Analysis}. Wiley.\\ 
\end{minipage}

Three \emph{row groups} have been made since there are three unique
categories under \texttt{groupname}. Across the top of each \emph{row
group} is the \emph{row group label} contained in a separate row (these
cut across the field and they contain nothing but the \emph{row group
label}). A rearrangement of rows is carried out to ensure each of the
rows is collected within the appropriate \emph{row groups}.

Having groups of rows in \emph{row groups} is a great way to present
information. Including data summaries particular to each group is a
natural extension of this idea. This process of adding summary rows with
\emph{summary labels} is covered in a separate article (\emph{Creating
Summary Lines}).

Another way to make row groups is to have the magic column
\texttt{groupname} present in the input data table. For our above
example with \texttt{islands\_tbl}, a \texttt{groupname} column with the
categories \texttt{continent}, \texttt{country}, and \texttt{subregion}
in the appropriate rows would produce row groups automatically (i.e.,
there would be no need to use the \texttt{tab\_row\_group()}
statements). This strategy of supplying group names in a
\texttt{groupname} column name can sometimes be advantageous since we
can rely on functions such as those available in \textbf{dplyr} to
generate the categories (e.g., using \texttt{case\_when()} or
\texttt{if\_else()}).

\subsubsection{The Column Labels}\label{the-column-labels}

The table's \textbf{Column Labels} part contains, at a minimum, columns
and their \emph{column labels}. The last example had a single column:
\texttt{size}. Just as in the \textbf{Stub}, we can create groupings
called \emph{spanner columns} that encompass one or more columns.

To better demonstrate how \textbf{Column Labels} work and are displayed,
let's use an input data table with more columns. In this case, that
input table will be \texttt{airquality}. It has the following columns:

\begin{itemize}
\tightlist
\item
  \texttt{Ozone}: mean ground-level ozone in parts per billion by volume
  (ppbV), measured between 13:00 and 15:00
\item
  \texttt{Solar.R}: solar radiation in Langley units (cal/m2), measured
  between 08:00 and noon
\item
  \texttt{Wind}: mean wind speed in miles per hour (mph)
\item
  \texttt{Temp}: maximum daily air temperature in degrees Fahrenheit
  (°F)
\item
  \texttt{Month}, \texttt{Day}: the numeric month and day of month for
  the record
\end{itemize}

We know that all measurements took place in 1973, so a \texttt{year}
column will be added to the dataset before it is passed to
\texttt{gt()}.

Let's organize the time information under a \texttt{Time} \emph{spanner
column label}, and put the other columns under a \texttt{Measurement}
\emph{spanner column label}. We can do this with the
\texttt{tab\_spanner()} function.

\begin{Shaded}
\begin{Highlighting}[]
\CommentTok{# Modify the `airquality` dataset by adding the year}
\CommentTok{# of the measurements (1973) and limiting to 10 rows}
\NormalTok{airquality_m <-}\StringTok{ }
\StringTok{  }\NormalTok{airquality }\OperatorTok
\StringTok{  }\KeywordTok{mutate}\NormalTok{(}\DataTypeTok{Year =}\NormalTok{ 1973L) }\OperatorTok
\StringTok{  }\KeywordTok{slice}\NormalTok{(}\DecValTok{1}\OperatorTok{:}\DecValTok{10}\NormalTok{)}
  
\CommentTok{# Create a display table using the `airquality`}
\CommentTok{# dataset; arrange columns into groups}
\NormalTok{gt_tbl <-}\StringTok{ }
\StringTok{  }\KeywordTok{gt}\NormalTok{(}\DataTypeTok{data =}\NormalTok{ airquality_m) }\OperatorTok
\StringTok{  }\KeywordTok{tab_header}\NormalTok{(}
    \DataTypeTok{title =} \StringTok{"New York Air Quality Measurements"}\NormalTok{,}
    \DataTypeTok{subtitle =} \StringTok{"Daily measurements in New York City (May 1-10, 1973)"}
\NormalTok{  ) }\OperatorTok
\StringTok{  }\KeywordTok{tab_spanner}\NormalTok{(}
    \DataTypeTok{label =} \StringTok{"Time"}\NormalTok{,}
    \DataTypeTok{columns =} \KeywordTok{vars}\NormalTok{(Year, Month, Day)}
\NormalTok{  ) }\OperatorTok
\StringTok{  }\KeywordTok{tab_spanner}\NormalTok{(}
    \DataTypeTok{label =} \StringTok{"Measurement"}\NormalTok{,}
    \DataTypeTok{columns =} \KeywordTok{vars}\NormalTok{(Ozone, Solar.R, Wind, Temp)}
\NormalTok{  )}
\CommentTok{# Show the gt Table}
\NormalTok{gt_tbl}
\end{Highlighting}
\end{Shaded}

\captionsetup[table]{labelformat=empty,skip=1pt}

\begin{longtable}{rrrrrrr}
\caption*{
\large New York Air Quality Measurements\\ 
\small Daily measurements in New York City (May 1-10, 1973)\\ 
} \\ 
\toprule
\multicolumn{4}{c}{Measurement} & \multicolumn{3}{c}{Time} \\ 
 \cmidrule(lr){1-4}\cmidrule(lr){5-7}
Ozone & Solar.R & Wind & Temp & Month & Day & Year \\ 
\midrule
41 & 190 & 7.4 & 67 & 5 & 1 & 1973 \\ 
36 & 118 & 8.0 & 72 & 5 & 2 & 1973 \\ 
12 & 149 & 12.6 & 74 & 5 & 3 & 1973 \\ 
18 & 313 & 11.5 & 62 & 5 & 4 & 1973 \\ 
NA & NA & 14.3 & 56 & 5 & 5 & 1973 \\ 
28 & NA & 14.9 & 66 & 5 & 6 & 1973 \\ 
23 & 299 & 8.6 & 65 & 5 & 7 & 1973 \\ 
19 & 99 & 13.8 & 59 & 5 & 8 & 1973 \\ 
8 & 19 & 20.1 & 61 & 5 & 9 & 1973 \\ 
NA & 194 & 8.6 & 69 & 5 & 10 & 1973 \\ 
\bottomrule
\end{longtable}

We can do two more things to make this presentable:

\begin{itemize}
\tightlist
\item
  move the \texttt{Time} columns to the beginning of the series (using
  \texttt{cols\_move\_to\_start()})
\item
  customize the column labels so that they are more descriptive (using
  \texttt{cols\_label()})
\end{itemize}

Let's do both of these things in the next example.

\begin{Shaded}
\begin{Highlighting}[]
\CommentTok{# Move the time-based columns to the start of}
\CommentTok{# the column series; modify the column labels of}
\CommentTok{# the measurement-based columns}
\NormalTok{gt_tbl <-}\StringTok{ }
\StringTok{  }\NormalTok{gt_tbl }\OperatorTok
\StringTok{  }\KeywordTok{cols_move_to_start}\NormalTok{(}
    \DataTypeTok{columns =} \KeywordTok{vars}\NormalTok{(Year, Month, Day)}
\NormalTok{  ) }\OperatorTok
\StringTok{  }\KeywordTok{cols_label}\NormalTok{(}
    \DataTypeTok{Ozone =} \KeywordTok{html}\NormalTok{(}\StringTok{"Ozone,<br>ppbV"}\NormalTok{),}
    \DataTypeTok{Solar.R =} \KeywordTok{html}\NormalTok{(}\StringTok{"Solar R.,<br>cal/m<sup>2</sup>"}\NormalTok{),}
    \DataTypeTok{Wind =} \KeywordTok{html}\NormalTok{(}\StringTok{"Wind,<br>mph"}\NormalTok{),}
    \DataTypeTok{Temp =} \KeywordTok{html}\NormalTok{(}\StringTok{"Temp,<br>&deg;F"}\NormalTok{)}
\NormalTok{  )}
\CommentTok{# Show the gt Table}
\NormalTok{gt_tbl}
\end{Highlighting}
\end{Shaded}

\captionsetup[table]{labelformat=empty,skip=1pt}

\begin{longtable}{rrrrrrr}
\caption*{
\large New York Air Quality Measurements\\ 
\small Daily measurements in New York City (May 1-10, 1973)\\ 
} \\ 
\toprule
\multicolumn{3}{c}{Time} & \multicolumn{4}{c}{Measurement} \\ 
 \cmidrule(lr){1-3}\cmidrule(lr){4-7}
Year & Month & Day & Ozone,<br>ppbV & Solar R.,<br>cal/m<sup>2</sup> & Wind,<br>mph & Temp,<br>&deg;F \\ 
\midrule
1973 & 5 & 1 & 41 & 190 & 7.4 & 67 \\ 
1973 & 5 & 2 & 36 & 118 & 8.0 & 72 \\ 
1973 & 5 & 3 & 12 & 149 & 12.6 & 74 \\ 
1973 & 5 & 4 & 18 & 313 & 11.5 & 62 \\ 
1973 & 5 & 5 & NA & NA & 14.3 & 56 \\ 
1973 & 5 & 6 & 28 & NA & 14.9 & 66 \\ 
1973 & 5 & 7 & 23 & 299 & 8.6 & 65 \\ 
1973 & 5 & 8 & 19 & 99 & 13.8 & 59 \\ 
1973 & 5 & 9 & 8 & 19 & 20.1 & 61 \\ 
1973 & 5 & 10 & NA & 194 & 8.6 & 69 \\ 
\bottomrule
\end{longtable}

Note that even though columns were moved using
\texttt{cols\_move\_to\_start()}, the \emph{spanner column labels} still
spanned above the correct \emph{column labels}. There are a number of
functions that \textbf{gt} provides to move columns, including
\texttt{cols\_move()}, \texttt{cols\_move\_to\_end()}; there's even a
function to hide columns: \texttt{cols\_hide()}.

Multiple columns can be renamed in a single use of
\texttt{cols\_label()}. Further to this, the helper functions
\texttt{md()} and \texttt{html()} can be used to create column labels
with additional styling. In the above example, we provided column labels
as HTML so that we can insert linebreaks with
\texttt{\textless{}br\textgreater{}}, insert a superscripted \texttt{2}
(with
\texttt{\textless{}sup\textgreater{}2\textless{}/sup\textgreater{}}),
and insert a degree symbol as an HTML entity (\texttt{\&deg;}).


\end{document}
